  \documentclass{jsarticle}
 
 \title{VISUALIZING SCALE FREE NETWORKS FOR THE ANALYSIS OF P2P NETWORK TRAFFIC DATA\\ P2Pネットワーク解析のためのスケールフリーネットワークの可視化}
 \author{by \\ Kai Sasaki \\ 佐々木海}
 \date{}
 \begin{document}
 \maketitle
 \begin{center} \Large{A Senior Thesis \\ 卒業論文}\end{center}
\begin{verbatim}








\end{verbatim}

 \begin{center} \Large{Submitted to \\ the Department of Information Science \\ on February 7,2011\\ in partial fulfilment of the requirements \\ for the Degree of Bachelor of Science}\end{center}
 \begin{verbatim}
 
 \end{verbatim}
 
 \begin{center}
 \Large{Thesis Supervisor : Shigeo Takahashi 高橋 成雄\\ Title : Associate Professor of Information Science}
 \end{center}
 
 \newpage
\begin{center} \LARGE{ABSTRACT} \end{center}
This thesis presents a method for visualizing scale-free networks in order to analyze the P2P network traffic data. P2P is a networking architecture used in various fields such as file sharing systems, telephone systems,and broadcasting , and its networking topology often constitutes a scale-free network where the vertex degrees vary at an exponential rate. Visualizing P2P network traffic paths by drawing the corresponding network structure is helpful for us to explore better network service as well as to maintain the security of the network itself. In this thesis, the nodes of the graph represent Peers and autonomous systems (ASs) while its edges correspond to network paths along which files are
transferred. The P2P network structure has been embedded onto the 2D plane employing conventional graph drawing algorithms such as the spring model
and multi-dimensional scaling techniques. The traffic datasets to be analyzed in this thesis are network paths of file transfers with Winny and BitTorrent, which have been obtained by monitoring core hubs of the network. We obtained several visualization results that are presented to demonstrate that we can visually understand the connectivity between Peers and ASs of each swarm together with the amount of transferred data.
\\


\begin{center} \LARGE{論文要旨} \end{center}
本論文ではP2Pネットワークのトラフィックデータを解析するためのスケールフリーなネットワークの可視化手法について述べる。P2Pはファイル共有、電話、放送など様々な分野で利用されているネットワークアーキテクチャであり、そのネットワークトポロジーは頂点の次数が指数関数的に増減するようなスケールフリーなネットワークを構成する。P2Pネットワークのトラフィックパスを通信ネットワークの構造を用いて可視化することはネットワークの安全性を保つことと同様に、よりよいネットワークサービスを模索する上でも有用である。本手法ではグラフのノードはPeerまたはASを表し、エッジはファイル送受信の軌跡を表す。P2Pネットワークの構造は一般的に用いられているspring modelやMDSなどの手法を用いて平面グラフとして可視化した。本手法で分析されたトラフィックデータはネットワーク上に配置されたコアハブによる監視で得られたWinnyとBitTorrentのファイル送受信の軌跡データである。それぞれのSwarmのPeerやASの接続性を送受信データによって視覚的に理解できることを示す事例を幾つか得る事ができた。
 \end{document}